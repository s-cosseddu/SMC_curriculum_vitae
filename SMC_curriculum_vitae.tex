\documentclass[a4paper,10pt,final]{memoir}

% forked from 
% http://www.howtotex.com/templates/creating-a-designers-cv-in-latex/

% common packages
\usepackage{graphicx}
\usepackage{url}
\usepackage{color}
\usepackage[usenames,dvipsnames]{xcolor}

% ####################################
%            page style
% ###################################
\renewcommand{\familydefault}{bch}
\pagestyle{empty}
\setlength{\parindent}{0pt}
% margins
\usepackage[top=2.0cm,left=1.9cm,right=1.9cm,bottom=2.0cm]{geometry}


% ####################################
%               lists 
% ###################################
% compact and columned
\usepackage{multicol}
	\setlength{\multicolsep}{0pt}
\usepackage{paralist}
% change symbol standard list
\usepackage{enumitem}
\renewcommand{\labelitemi}{\color{RoyalBlue}$\circ$}


%%%%%%%%%%%%%%%%%%%%%%%%%%%%%%%%%%%%%
% define macros 
%%%%%%%%%%%%%%%%%%%%%%%%%%%%%%%%%%%%%

\newcommand{\Sep}{\vspace{1.5em}}
\newcommand{\SmallSep}{\vspace{0.5em}}

%'About me' section:
%environment is called AboutMe and can be used with \begin{AboutMe}...\end{AboutMe}.
\newenvironment{AboutMe}
	{\ignorespaces\textbf{\color{RoyalBlue} About me}}
	{\Sep\ignorespacesafterend}

% to create a new section (\CVSection) and to create a new item/entry (\CVItem).

\newcommand{\CVSection}[1]
	{\Large\textbf{#1}\par
	\SmallSep\normalsize\normalfont}
\newcommand{\CVItem}[2]
	{\textit{\textbf{\color{RoyalBlue} #1}} #2}


%######################## 
%        MAIN      
%######################## 
\begin{document}

%----------------------------------------------------------------------------------------
%	YOUR NAME AND ADDRESS(ES) SECTION
%----------------------------------------------------------------------------------------
\Huge\bfseries {\color{RoyalBlue} Salvatore M. Cosseddu} \\
%\Large\bfseries  Graphics designer \\
\small
Centre of Scientific Computing \\
Computer Science Building \\
University of Warwick \\
Coventry, CV4 7AL, United Kingdom \\
\url{S.M.Cosseddu@warwick.ac.uk}
%----------------------------------------------------------------------------------------
\normalsize\normalfont
\SmallSep
\Sep

\CVSection{EDUCATION}
\CVItem{ Doctor of Philosophy}{(PhD) Biophysics/Biophysical Chemistry \hfill April 2010 to date} \\
School of Engineering and Centre for Scientific Computing, \\
University of Warwick, Coventry, UK %\\
\begin{description} [style=multiline,leftmargin=3cm,font=\normalfont] \itemsep -2pt
\item [Thesis:] \textit{Structure and Dynamics of Protein in the Permeation and Gating of
  Potassium Ion Channels: Identifying Molecular Determinants and Developing Coarse-Grained
  Approaches}.
\item [Research description:] The project has been developed under the supervision of Dr
  Igor Khovanov (School of Engineering), Prof Mike P Allen (Department of Physics) and
  Prof Mark Rodger (Department of Chemistry). The research aimed to investigate the
  inactivation process in potassium ion channel KcsA and its relation with the ions
  permeation using fine atomistic models such as Molecular Dynamics and free-energy
  methods (Metadynamics and Umbrella Sampling). The study was focused on the dynamics of
  the filter and the nearby network of residues able to regulate the conductivity. An
  additional goal was to further develop the outcomes into coarse-grained Brownian dynamics
  models. 
\end{description}
\SmallSep

\CVItem{Master of Science}{Physical Chemistry and Inorganic Chemistry (MSc, First-Class Honours)\hfill March 2010}\\
Universit\`a degli Studi di Sassari, Sassari, Italy %\\ 
\begin{description} [style=multiline,leftmargin=3cm,font=\normalfont] \itemsep -2pt
\item [Thesis:] One year project on developing a Kinetic Monte Carlo model to describe
  diffusion and reactivity in MFI-type zeolites such as Silicalite-1 and ZSM-5, in
  particular xylene
  isomerization on HZSM-5.
\item[Main subjects:] Physical chemistry, statistical mechanics, solid state chemistry,
  pharmaceutical chemistry, inorganic chemistry, asymmetric catalysis, aromatic compounds
  chemistry. 
\end{description}
\SmallSep
 
\CVItem{Bachelor of Science}{Chemistry (BSc, First-Class Honours) \hfill  March 2008}\\
Universit\`a degli Studi di Sassari, Sassari, Italy %\\
\begin{description} [style=multiline,leftmargin=3cm,font=\normalfont] \itemsep -2pt
\item [Thesis:] Six month project on developing a Cellular Lattice Gas Automaton to investigate
the reactivity of molecules diffusing in zeolites.
\item[Main subjects:] Physical chemistry, organic chemistry, analytical chemistry,
  inorganic chemistry, biochemistry, industrial chemistry and polymers, mathematics,
  physics and statistics.
\end{description}


%----------------------------------------------------------------------------------------
%	PROFESSIONAL EXPERIENCE SECTION
%----------------------------------------------------------------------------------------
\Sep
\CVSection{PROFESSIONAL EXPERIENCE} 
\CVItem{Laboratory demonstrator} \\ %{\hfill Apr 2010 - Present} \\
University of Warwick, Coventry, UK  \\
Laboratory demonstrator for Material Microstructure Laboratory and Statistical Mechanics. 

%----------------------------------------------------------------------------------------
%	Additional Training and Conferences
%----------------------------------------------------------------------------------------
\Sep
\CVSection{ADDITIONAL TRAINING AND CONFERENCES} 
\CVItem{Additional Training} 
\begin{itemize} \itemsep -2pt % Reduce space between items
\item \textit{CCP-BioSim workshop on Free energy methods for modelling of protein-ligand
  interactions}, 21 Nov 2012, University of Southampton, UK;
\item \textit{An Introduction to NAG Numerical Components}, 10 Jul 2012, University of Warwick, UK;
\item \textit{CECAM/TCBG Computational Biophysics Workshop in Bremen},  17 - 21 Oct 2011, Jacobs University, Germany;  
\item \textit{CCP5 DL\_POLY Training Workshop}, 2 - 3 Feb 2011, Daresbury
Laboratory, UK;
\item \textit{CSC / NAG Debugging, Profiling and Optimising},  8 - 9 Nov 2011, University of
  Warwick, UK;
\item \textit{CSC / NAG Autumn School in Core Algorithms for High Performance Scientific
  Computing}, 26 - 30 Sep 2011, University of Warwick, UK;
\item \textit{High Performance Scientific Computing} module, 2010/2011, University of Warwick, UK;
\item \textit{Monte Carlo and molecular dynamics} module, 2010/2011, MPAGS, University of Warwick, UK;
\item \textit{CCP5 CECAM Methods in Molecular Simulation Summer School 2010}, 18 - 27 Jul 2010, Queens University
  Belfast, UK.
\end{itemize}
\SmallSep

\CVItem{Conferences} 
\begin{itemize} \itemsep -2pt  % Reduce space between items
\item \textit{CCP5/RSC workshop Advances in Theory and Simulation of non-Equilibrium Systems},
  26 - 27 Jun 2013, Imperial College, UK;
\item \textit{CCP5-MDNet Mathematical Challenges in Molecular Dynamics}, 2 - 5 Apr 2013, University
  of Warwick, UK; 
\item Poster presented to \textit{2nd Annual CCP-BioSim Conference, Frontiers of Biomolecular
  Simulation}, 25 - 27 Mar 2013, University of Nottingham, UK;
\item Poster presented to \textit{Beyond Molecular Dynamics: Long Time Atomic-Scale Simulations}, 26 - 29 Mar 2012,
  Max Planck Institute for the Physics of Complex Systems, Dresden, Germany.
\item \textit{CCPB Collective Variable Methods in Biomolecular Simulation Principles and Applications},
4 Nov 2011, University of Nottingham, UK
\item \textit{Mathematical Modelling of Ion Channels Workshop}, 5 - 6 Sep 2011, St Anne's
  College, Oxford, UK;
\item \textit{Trends in protein biophysics: from in silico molecules to in vivo and vitro
  proteins}, 17 - 19 May 2011, University of Warwick, UK;
\item \textit{MIRaW day on Monte Carlo Methods}, 7 Mar 2011, Mathematics Institute, University
  of Warwick, UK;
\item \textit{IOP Condensed Matter and Materials Physics CMMP10}, 14 -16 Dec 2010, University of Warwick, UK.
\end{itemize}

\Sep
\CVSection{PRIZES} 
Awarded with the prize for the best talk at Centre for Scientific Computing's postgraduate day 2012.


%----------------------------------------------------------------------------------------
%	 SKILLS SECTION
%----------------------------------------------------------------------------------------
\Sep
\CVSection{COMPUTATIONAL SKILLS}
\begin{multicols}{3}
\CVItem{
Programming languages and \\
tools\hfill
} \\
Excellent knowledge:
\begin{compactitem}[\color{RoyalBlue}$\circ$]
\item FORTRAN,
\item Tcl,
\item Bash.
\end{compactitem}
Good knowledge:
\begin{compactitem}[\color{RoyalBlue}$\circ$]
\item Make,
\item git.
\end{compactitem}
Basic knowledge:
\begin{compactitem}[\color{RoyalBlue}$\circ$]
\item C,
\item Matlab,
\item Python,
\item profilers.
\end{compactitem}
\SmallSep
 
\CVItem{Tools for molecular simulations\\ and statistical analysis:\hfill}
\begin{compactitem}[\color{RoyalBlue}$\circ$]
\item NAMD,
\item VMD,
\item R,
\item gnuplot.
\end{compactitem}
\SmallSep

\CVItem{Office tools:\hfill}
\begin{compactitem}[\color{RoyalBlue}$\circ$]
\item LaTeX,
\item Emacs,
\item Microsoft Office,
\item OpenOffice/LibreOffice. 
\end{compactitem}
\SmallSep

\CVItem{System administrator:\hfill}
\begin{compactitem}[\color{RoyalBlue}$\circ$]
\item GNU/Linux (Debian and derived, Fedora, SUSE),
\item Mac OS X,
\item Windows OS.
\end{compactitem}
\SmallSep

\CVItem{Image manipulation programs\\ and 2d and 3d graphics editor:\hfill}
\begin{compactitem}[\color{RoyalBlue}$\circ$]
\item Gimp,
\item Inkscape,
\item Blender.
\end{compactitem}

\end{multicols}


%----------------------------------------------------------------------------------------


%----------------------------------------------------------------------------------------
%	PUBLICATIONS SECTION
%----------------------------------------------------------------------------------------
\Sep
\CVSection{PUBLICATIONS} 
\begin{itemize} \itemsep -2pt % Reduce space between items
\item S. M. Cosseddu, I. A. Khovanov, M. P. Allen, P. M. Rodger, D. G. Luchinsky,
  P. V. E. McClintock, \textit{Dynamics of Ions in the Selectivity Filter of the KcsA Channel: Towards a Coupled
  Brownian Particle Description}, EJP-ST, 2013, accepted.
\item S. M. Cosseddu, M. P. Allen, P. M. Rodger, I. A. Khovanov, \textit{Mechanism and Energetic
  of C-type Inactivation in K$^+$ Ion Channels}, in preparation.
\end{itemize}


%----------------------------------------------------------------------------------------
%	MEMBERSHIPS SECTION
%----------------------------------------------------------------------------------------
\Sep
\CVSection{MEMBERSHIPS} 
Associate member of the Institute of Physics. 

%----------------------------------------------------------------------------------------
%	HONORS SECTION
%----------------------------------------------------------------------------------------
\Sep
\CVSection{LANGUAGES} 
\begin{itemize} \itemsep -2pt % Reduce space between items
\item Italian - native language.
\item English - speak fluently and read/write with high proficiency.
\end{itemize}
% ----------------------------------------------------------------------------------------

\end{document}