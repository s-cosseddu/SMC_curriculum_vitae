\documentclass[a4paper,10pt,final]{memoir}

% forked from 
% http://www.howtotex.com/templates/creating-a-designers-cv-in-latex/

% common packages
\usepackage{graphicx}
\usepackage{url}
\usepackage{color}
\usepackage[usenames,dvipsnames]{xcolor}

% ####################################
%            page style
% ###################################
\renewcommand{\familydefault}{bch}
\usepackage{microtype}
\pagestyle{empty}
\setlength{\parindent}{0pt}
% margins
\usepackage[top=2.0cm,left=1.9cm,right=1.9cm,bottom=2.0cm]{geometry}


% ####################################
%               lists 
% ###################################
% compact and columned
\usepackage{multicol}
	\setlength{\multicolsep}{0pt}
\usepackage{paralist}
% change symbol standard list
\usepackage{enumitem}
\renewcommand{\labelitemi}{\color{RoyalBlue}$\circ$}


%%%%%%%%%%%%%%%%%%%%%%%%%%%%%%%%%%%%%
% define macros 
%%%%%%%%%%%%%%%%%%%%%%%%%%%%%%%%%%%%%

\newcommand{\Sep}{\vspace{1em}}
\newcommand{\SmallSep}{\vspace{0.4em}}

%'About me' section:
%environment is called AboutMe and can be used with \begin{AboutMe}...\end{AboutMe}.
\newenvironment{AboutMe}
	{\ignorespaces\textbf{\color{RoyalBlue} About me}}
	{\Sep\ignorespacesafterend}

% to create a new section (\CVSection) and to create a new item/entry (\CVItem).

\newcommand{\CVSection}[1]
	{\Large\textbf{#1}\par
	\SmallSep\normalsize\normalfont}
\newcommand{\CVItem}[2]
	{\textit{\textbf{\color{RoyalBlue} #1}} #2}


%######################## 
%        MAIN      
%######################## 
\begin{document}

%----------------------------------------------------------------------------------------
%	YOUR NAME AND ADDRESS(ES) SECTION
%----------------------------------------------------------------------------------------
\Huge {\bfseries \color{RoyalBlue} Salvatore M. Cosseddu} \\
\small
 \textbf{E-Mail:} \url{salvatore.cosseddu@gmail.com} \\
\textbf{Telephone:} +44 (0)77 38245847
%----------------------------------------------------------------------------------------
\normalsize\normalfont
\SmallSep
\Sep

\CVSection{EDUCATION}
\CVItem{ Doctor of Philosophy}{Biophysics/Biophysical Chemistry (PhD) \hfill April 2010 to date} \\
School of Engineering and Centre for Scientific Computing, \\
University of Warwick, Coventry, UK %\\
\begin{description} [style=multiline,leftmargin=2.6cm,font=\normalfont,topsep=0.1cm,itemsep=-3.5pt]
\item [Thesis:] \textit{Structure and Dynamics of Protein in the Permeation, Selectivity
    and Gating of Potassium Ion Channels}.
\item [Research description:] Interdisciplinary joint research project between the
  Universities of Warwick and Lancaster, developed under the supervision of Dr Igor
  Khovanov (School of Engineering), Prof Mike P Allen (Department of Physics) and Prof
  Mark Rodger (Department of Chemistry), in collaboration with the Prof McClintock's group
  (Department of Physics, University of Lancaster), funded by EPSRC. I was associated with
  ``Molecular Simulation'' and ``Biomedical'' research groups. The research covered
  different fundamental features of K$^+$ ion channels: permeation, selectivity and
  inactivation processes. The research took advantage of the large-scale high performance
  computing resources of the Centre for Scientific Computing (University of
  Warwick).% ($\sim 6000$ core 4X QDR InfiniBand Linux cluster).
\end{description}
\SmallSep

\CVItem{Master of Science}{Physical Chemistry and Inorganic Chemistry (MSc, First-Class Honours)\hfill March 2010}\\
Universit\`a degli Studi di Sassari, Sassari, Italy %\\ 
\begin{description}[style=multiline,leftmargin=2.6cm,font=\normalfont,topsep=0.1cm,itemsep=-3.5pt]
\item [Final project:] One year project on developing a Kinetic Monte Carlo model to
  describe diffusion and reactivity in MFI-type zeolites.
\end{description}
  \SmallSep
 
\CVItem{Bachelor of Science}{Chemistry (BSc, First-Class Honours) \hfill  March 2008}\\
Universit\`a degli Studi di Sassari, Sassari, Italy\\
University of Nottingham, Nottingham, UK, six months within the Erasmus exchange programme  %\\
\begin{description}[style=multiline,leftmargin=2.6cm,font=\normalfont,topsep=0.1cm,,itemsep=-3.5pt]
\item [Final project:] Six-months project on a Cellular Lattice Gas Automaton model to simulate
  reactivity in zeolites.
\end{description}


%----------------------------------------------------------------------------------------
%	PROFESSIONAL EXPERIENCE SECTION
%----------------------------------------------------------------------------------------
\Sep
\CVSection{PROFESSIONAL EXPERIENCE}
\CVItem{Researcher} \\
Strong experience on investigation of 
\begin{compactitem}[\color{RoyalBlue}$\circ$] 
\item Mechanisms and energetics of biological and non-biological processes;
\item Structure-function relationship of protein;
\item Highly-correlated non-linear dynamical properties of biological systems;
\item Affinities between proteins and their substrates.
\end{compactitem}

I have been involved in a joint research proposal on biological ion channels, submitted to
EPSRC, between universities of Warwick and Lancaster in which I am a named researcher, PI
Prof P.V.E. McClintock.

\SmallSep
\CVItem{Laboratory demonstrator} \\ %{\hfill Apr 2010 - Present} \\
University of Warwick, Coventry, UK  \\
Laboratory demonstrator for Material Microstructure Laboratory and Statistical Mechanics. 


%----------------------------------------------------------------------------------------
%	 SKILLS SECTION
%----------------------------------------------------------------------------------------
\Sep
\CVSection{SCIENTIFIC/TECHNICAL SKILLS}

\CVItem{Broad interdisciplinary background} 
\begin{multicols}{3}
\begin{compactitem}[\color{RoyalBlue}$\circ$]
\item Physical chemistry,
\item Biophysics,
\item Biochemistry,
\item Molecular biology,
\item Organic and inorganic chemistry,
\item Pharmaceutical chemistry,
\item Statistics,
\item Classical and quantum mechanics
\item Statistical mechanics,
\item Thermodynamics,
\item Scientific and high performance computing. 
\end{compactitem}
\end{multicols}
\SmallSep
\CVItem{Technical skills}
\begin{multicols}{3}
\begin{compactitem}[\color{RoyalBlue}$\circ$]
\item Fine atomistic models (MD),
\item State-of-the-art free-energy\\ methods (MetaD, US)
\item Statistical analyses.
\item Coarse grained models (MC, KMC, BD)
\end{compactitem}
\end{multicols}
\Sep
\pagebreak
% other skills
\CVSection{COMPUTATIONAL SKILLS}
\begin{multicols}{3}
\CVItem{Molecular simulations\\ and statistical analysis}
\begin{compactitem}[\color{RoyalBlue}$\circ$]
\item NAMD,
\item VMD,
\item R,
\item Gnuplot.
\end{compactitem}
\SmallSep 
\CVItem{Languages and 
  tools}\\
Excellent knowledge:
\begin{compactitem}[\color{RoyalBlue}$\circ$]
\item FORTRAN,
\item Tcl,
\item Bash,
\end{compactitem}
Additional:
\begin{compactitem}[\color{RoyalBlue}$\circ$]
\item Make and git (good),
\item C, Matlab, Python (basic).
\end{compactitem}
\SmallSep
\CVItem{Office and graphics tools} 
\begin{compactitem}[\color{RoyalBlue}$\circ$]
\item \LaTeX{} (including Beamer),
\item Emacs,
\item Microsoft Office/OpenOffice, 
\item Gimp, Inkscape and Blender.
\end{compactitem}
\SmallSep

\CVItem{System administrator}
\begin{compactitem}[\color{RoyalBlue}$\circ$]
\item GNU/Linux (Debian and derived, Fedora, SUSE; I personally managed the Debian
  GNU/Linux workstations used during my PhD and MSc projects),
\item Mac OS X,
\item Windows OS.
\end{compactitem}
\end{multicols}

%----------------------------------------------------------------------------------------
% compactitem slightly wider
%\setlength{\plitemsep}{0.15pt}
%\setlength{\pltopsep}{2.5pt}
%\setlength{\topsep}{-1cm}

\Sep
\CVSection{COMMUNICATION SKILLS}
\begin{compactitem}[\color{RoyalBlue}$\circ$]
\item Pitch presentation and Poster presented to \textit{2nd Annual CCP-BioSim Conference, Frontiers of
    Biomolecular Simulation}, 25 - 27 Mar 2013, Uni.\,of Nottingham, UK;
\item Presentation of my research on \emph{inactivation} in K$^+$ channels at
  CSC seminar, 11 Mar 2013, Uni.\,of Warwick, UK;
\item Presentation of my research on \emph{permeation} in K$^+$  channels at CSC
  seminar, 2 Dec 2013, Uni.\,of Warwick, UK;
\item Presentations at postgraduate days of both CSC and School of Engineering;
\item Poster presented to \textit{South West Computational Chemists annual
    meeting 2013}, 24 Sep 2013, Uni.\,of Southampton, UK;
\item Poster presented to \textit{Beyond Molecular Dynamics: Long Time Atomic-Scale Simulations}, 26 - 29 Mar 2012,
  Max Planck Institute for the Physics of Complex Systems, Dresden, Germany;
\item I discussed and presented my works in meetings with various collaborators in which I
  learnt how to communicate with people with very different backgrounds;
\item I presented my research and published papers in several occasions during regular
  meetings of the groups I was associated with;
\end{compactitem}

\Sep
\CVSection{PRIZES} 
\begin{compactitem}[\color{RoyalBlue}$\circ$]
\item Awarded with the prize for the best talk at Centre for Scientific Computing's postgraduate day 2012.
\item Poster selected by the School of Engineering in 2011 (still exposed) for advertising the
  different research activities of the department.
\end{compactitem}

%----------------------------------------------------------------------------------------
%	Additional Training and Conferences
%----------------------------------------------------------------------------------------

\Sep
\CVSection{SELECTED ADDITIONAL CONFERENCES AND TRAINING} 

\begin{compactitem}[\color{RoyalBlue}$\circ$]
\item \textit{CCP5/RSC workshop Advances in Theory and Simulation of non-Equilibrium Systems},
  26 - 27 Jun 2013, Imperial College, UK;
\item \textit{CCP-BioSim workshop on Free energy methods for modelling of protein-ligand
  interactions}, 21 Nov 2012, Uni.\,of Southampton, UK;
\item \textit{Mathematical Modelling of Ion Channels Workshop}, 5 - 6 Sep 2011, St Anne's
  College, Oxford, UK;
\item \textit{Trends in protein biophysics: from in silico molecules to in vivo and vitro
  proteins}, 17 - 19 May 2011, Uni.\,of Warwick, UK;
\item \textit{CECAM/TCBG Computational Biophysics Workshop in Bremen},  17 - 21 Oct 2011,
  Jacobs University, Germany;  
\item \textit{IOP Condensed Matter and Materials Physics CMMP10}, 14 -16 Dec 2010,
  Uni.\,of Warwick, UK;
\item \textit{High Performance Scientific Computing} module, 2010/2011, Uni.\,of Warwick,
  UK;
\item \textit{CCP5 CECAM Methods in Molecular Simulation Summer School 2010}, 18 - 27 Jul 2010, Queens University
  Belfast, UK.
%\item \textit{CCP5-MDNet Mathematical Challenges in Molecular Dynamics}, 2 - 5 Apr 2013, University
%  of Warwick, UK; 
%\item \textit{CCPB Collective Variable Methods in Biomolecular Simulation Principles and Applications},
%4 Nov 2011, Uni.\,of Nottingham, UK
%\item \textit{MIRaW day on Monte Carlo Methods}, 7 Mar 2011, Mathematics Institute, University
%  of Warwick, UK;
%\item \textit{An Introduction to NAG Numerical Components}, 10 Jul 2012, Uni.\,of Warwick, UK;
%\item \textit{CCP5 DL\_POLY Training Workshop}, 2 - 3 Feb 2011, Daresbury
%Laboratory, UK;
%\item \textit{CSC / NAG Debugging, Profiling and Optimising},  8 - 9 Nov 2011, Uni.\,of Warwick, UK;
% \item \textit{CSC / NAG Autumn School in Core Algorithms for High Performance Scientific
%   Computing}, 26 - 30 Sep 2011, Uni.\,of Warwick, UK;
%\item \textit{Monte Carlo and molecular dynamics} module, 2010/2011, MPAGS, Uni.\,of Warwick, UK;
\end{compactitem}

%----------------------------------------------------------------------------------------
%	PUBLICATIONS SECTION
%----------------------------------------------------------------------------------------
\Sep
\CVSection{PUBLICATIONS} 
\begin{compactitem}[\color{RoyalBlue}$\circ$]
\item S. M. Cosseddu, I. A. Khovanov, M. P. Allen, P. M. Rodger, D. G. Luchinsky,
  P. V. E. McClintock, \textit{Dynamics of Ions in the Selectivity Filter of the KcsA
    Channel: Towards a Coupled Brownian Particle Description}, EJP-ST, 222, 2595-2605,
  2013.
\item S. M. Cosseddu, M. P. Allen, P. M. Rodger, I. A. Khovanov, \textit{Highly-Coupled Network of
    Residues Underlying the Regulation of Conductivity in K$^+$ Ion Channels}, in preparation.
\item S. M. Cosseddu, M. P. Allen, P. M. Rodger, I. A. Khovanov, \textit{Mechanism and Energetic
  of C-type Inactivation in K$^+$ Ion Channels}, in preparation.
\item S. M. Cosseddu, M. P. Allen, P. M. Rodger, I. A. Khovanov, \textit{Energetics of
    Permeation and Selectivity of the Conductive State of K$^+$ Ion Channels}, in
  preparation.
\end{compactitem}

%----------------------------------------------------------------------------------------
%	MEMBERSHIPS SECTION
%----------------------------------------------------------------------------------------
\Sep
\CVSection{MEMBERSHIPS} 
Associate member of the Institute of Physics. 

% %----------------------------------------------------------------------------------------
% %	HONORS SECTION
% %----------------------------------------------------------------------------------------
% \Sep
% \CVSection{LANGUAGES} 
% %\begin{itemize} \itemsep -2pt % Reduce space between items
% \begin{compactitem}[\color{RoyalBlue}$\circ$]
% \item Italian - native language.
% \item English - speak fluently and read/write with high proficiency.
% \end{compactitem}
% % \end{itemize}
% % ----------------------------------------------------------------------------------------

\end{document}
